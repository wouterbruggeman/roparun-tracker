\documentclass[12pt, a4paper]{article}
\usepackage[utf8]{inputenc}
\usepackage[dutch]{babel}
\usepackage{hyperref}
\usepackage{color}
\usepackage{graphicx}
\usepackage{fancyhdr}
\usepackage{subfiles}
\usepackage{import}
\usepackage[subpreambles=true]{standalone}
% Skip to next paragraph on whiteline
\usepackage{parskip}

\graphicspath{{images}{../images/}}
\pagestyle{fancy}
\fancyhf{}
\lhead{Documentatie Roparun Tracker}
\rfoot{Pagina \thepage}
\date{\today}
\renewcommand{\contentsname}{Inhoudsopgave}

\definecolor{blue}{rgb}{0,0.5,1}
\hypersetup{
	colorlinks = true,
	linkcolor = blue,
	urlcolor = blue,
}

\title{\textbf{Documentatie Roparun Tracker}}
\author{
	Paul de Hek, 0941736 \\
	Rick van Vonderen, 0945444 \\
	Quentin Hoogwerf, 0929493 \\
	Wouter Bruggeman, 0943071 \\
}

\begin{document}
	\maketitle
	\thispagestyle{empty}
	\newpage

	\tableofcontents
	\setcounter{page}{1}
	\newpage

	\section{Inleiding}
	\import{sections/}{introduction}
	\newpage

	\section{Code eisen}
	\import{sections/}{code-requirements}
	\newpage

	\section{Project eisen}
	\import{sections/}{project-requirements}
	\newpage

	\section{Ontwerp}
	\import{sections/}{design}
	\newpage

	\section{Reflectie}
	\import{sections/}{reflection}
	\newpage
\end{document}
