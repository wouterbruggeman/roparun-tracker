\documentclass[../main.tex]{subfiles}
\begin{document}

In dit hoofdstuk staat beschreven wat de eisen aan de code zijn.

\subsection{Classes}
Het maken van classes is noodzakelijk omdat we Java gebruiken om de applicatie te ontwikkelen.
Classes maken het ook mogelijk om code her te gebruiken.

\subsection{Methodes}
Een methode van een classe mag en kan maar 1 ding doen.
Dit zorgt voor minder verwarring en duidelijkere methode namen.

\subsection{Commentaar}
Het gebruik van commentaar regels is een vereiste voor dit project. Een commentaar regel
moet duidelijk beschrijven waarom de onderstaande methodes worden uitgevoerd.

\subsection{Overig}
Om te voorkomen dat er onvrede ontstaat door het gebruiken van snakecase en CamelCase,
heb ik besloten om hier een advies over te geven.

Mijn advies is dan ook om gebruik te maken van CamelCase en alle methode en variable
namen in het Engels te schrijven. Dit zorgt ervoor dat de code ook goed leesbaar is voor
niet-nederlanders.

De code is te vinden op github:
\url{https://github.com/wouterbruggeman/roparun-tracker}

\end{document}
