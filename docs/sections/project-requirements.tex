\documentclass[../main.tex]{subfiles}
\begin{document}
	\begin{itemize}
		\item[--] Ik als supporter wil zien waar het team op dit moment aan het rennen is, zodat ik als globale
		prestatie van het team kan bekijken.
		\item[--] Ik als supporter wil dat de applicatie mij een ETA (estimated time of arrival) bij een bepaalde locatie
		kan geven, zodat ik mijn spandoek met aanmoediging op tijd neer kan zetten.
		\item[--] Ik wil als runner later op maps kunnen inzien welke stukken van het parcour ik gelopen heb, zodat ik
		een persoonlijk overzicht en geschiedenis heb.
		\item[--] Ik wil als runner binnen elk gelopen stuk mijn data zien: snelheid, hartslag, weghelling, etc., zodat ik
		mijn sportieve prestaties per gelopen stuk kan analyseren.
		\item[--] Ik wil als runner mijn totale statistieken kunnen zien, om zo mijn sportieve prestaties te vergelijken
		of nieuwe doelen te stellen.
		\item[--] Ik wil als gebruiker dat de applicatie zo energiezuinig mogelijk is, omdat de accu 72 uur mee moet
		gaan, want tijdens de rustmomenten heb ik belangrijkere zaken om te doen, zoals eten,
		massagestoel bemachtigen en slapen.
		\item[--] Ik wil als gebruiker dat de applicatie veilig is, zodat ongeautoriseerde mensen niet bij mijn
		persoonlijke data kunnen.
		\item[--] Ik wil als loper dat de applicatie zelf herkent dat ik aan het lopen ben (ipv op een knop drukken om
		“activiteit” te starten), omdat ik stijf van de adrenaline sta en het in 8 van de 10 keren vergeet en
		daarmee worden de statistieken onjuist.
	\end{itemize}
\end{document}
